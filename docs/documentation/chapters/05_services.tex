% Chapter 5: Services (Part 1)
\chapter{Service Layer}

\section{Overview}

The service layer contains business logic and orchestrates operations between controllers and repositories. Services are transactional and handle data transformation between entities and DTOs.

\section{AuthService}

\subsection{Purpose}
Handles authentication logic and JWT token generation. Simplified for demonstration with role-based access.

\subsection{Complete Source Code}

\begin{lstlisting}[caption=AuthService.java - Complete Implementation]
package com.quizforge.service;

import com.quizforge.dto.LoginRequest;
import com.quizforge.dto.LoginResponse;
import com.quizforge.model.User;
import com.quizforge.repository.UserRepository;
import com.quizforge.security.JwtUtil;
import org.springframework.beans.factory.annotation.Autowired;
import org.springframework.security.crypto.password.PasswordEncoder;
import org.springframework.stereotype.Service;

@Service
public class AuthService {

    @Autowired
    private UserRepository userRepository;

    @Autowired
    private PasswordEncoder passwordEncoder;

    @Autowired
    private JwtUtil jwtUtil;

    public LoginResponse login(LoginRequest request) {
        // Dummy authentication logic
        // admin@quizforge.com -> ADMIN role
        // any other email -> CANDIDATE role
        
        User.Role role;
        String name;
        
        if (''admin@quizforge.com''.equals(request.email())) {
            role = User.Role.ADMIN;
            name = ''Admin User'';
        } else {
            role = User.Role.CANDIDATE;
            name = ''Candidate User'';
        }
        
        String token = jwtUtil.generateToken(request.email(), 
                                             role.name());
        
        return new LoginResponse(token, request.email(), name, 
                                role.name());
    }
}
\end{lstlisting}

\subsection{Line-by-Line Explanation}

\begin{itemize}[leftmargin=*]
    \item \textbf{Line 12: @Service Annotation}
    \begin{itemize}
        \item Marks class as Spring service bean
        \item Component scanning auto-detection
        \item Enables transaction management
        \item Semantic indicator of service layer
    \end{itemize}
    
    \item \textbf{Line 15-22: Dependency Injection}
    \begin{itemize}
        \item \texttt{@Autowired}: Spring automatically injects dependencies
        \item \texttt{UserRepository}: For user database operations
        \item \texttt{PasswordEncoder}: BCrypt password hashing (unused in demo)
        \item \texttt{JwtUtil}: JWT token generation and validation
    \end{itemize}
    
    \item \textbf{Line 24: login Method Signature}
    \begin{itemize}
        \item Takes \texttt{LoginRequest} DTO with email and password
        \item Returns \texttt{LoginResponse} with token and user info
        \item Public method called by AuthController
    \end{itemize}
    
    \item \textbf{Line 25-27: Authentication Comment}
    \begin{itemize}
        \item Explains simplified authentication logic
        \item Production would verify against database
        \item Production would validate password hash
    \end{itemize}
    
    \item \textbf{Line 29-30: Variable Declaration}
    \begin{itemize}
        \item \texttt{role}: Will be ADMIN or CANDIDATE
        \item \texttt{name}: Display name for response
    \end{itemize}
    
    \item \textbf{Line 32-38: Role Assignment Logic}
    \begin{itemize}
        \item If email is ''admin@quizforge.com'', assign ADMIN role
        \item Otherwise, assign CANDIDATE role
        \item Simple demo logic for quick testing
        \item Real implementation would query database
    \end{itemize}
    
    \item \textbf{Line 40-41: Token Generation}
    \begin{itemize}
        \item Calls \texttt{jwtUtil.generateToken(email, role)}
        \item Creates signed JWT with email and role claims
        \item Token valid for duration specified in properties
        \item Used for subsequent authenticated requests
    \end{itemize}
    
    \item \textbf{Line 43-44: Response Creation}
    \begin{itemize}
        \item Creates \texttt{LoginResponse} DTO
        \item Includes: token, email, name, role
        \item Record type provides immutability
        \item Returned to controller, then client
    \end{itemize}
\end{itemize}

\subsection{Production Implementation Notes}

In a production system, the login method would:

\begin{enumerate}
    \item Query database for user by email
    \item Verify user exists
    \item Compare password hash using PasswordEncoder
    \item Check if account is active/not locked
    \item Log login attempt
    \item Handle failed login attempts
    \item Implement rate limiting
\end{enumerate}

\begin{lstlisting}[caption=Production Login Implementation Example]
public LoginResponse login(LoginRequest request) {
    User user = userRepository.findByEmail(request.email())
        .orElseThrow(() -> new BadCredentialsException(
            ''Invalid credentials''));
    
    if (!passwordEncoder.matches(request.password(), 
                                  user.getPassword())) {
        throw new BadCredentialsException(''Invalid credentials'');
    }
    
    String token = jwtUtil.generateToken(user.getEmail(), 
                                        user.getRole().name());
    
    return new LoginResponse(token, user.getEmail(), 
                            user.getName(), user.getRole().name());
}
\end{lstlisting}

\section{AdminService}

\subsection{Purpose}
Handles all admin operations including quiz CRUD operations and analytics.

\subsection{Dependencies}

\begin{lstlisting}[caption=AdminService Dependencies]
@Autowired
private QuizRepository quizRepository;

@Autowired
private QuestionRepository questionRepository;

@Autowired
private UserRepository userRepository;

@Autowired
private QuizAttemptRepository attemptRepository;
\end{lstlisting}

\subsection{Method: getAllQuizzes}

\begin{lstlisting}[caption=Get All Quizzes]
public List<QuizSummaryResponse> getAllQuizzes() {
    return quizRepository.findAll().stream()
            .map(this::toSummaryResponse)
            .collect(Collectors.toList());
}
\end{lstlisting}

\textbf{Explanation:}
\begin{itemize}
    \item Retrieves all quizzes from database
    \item Uses Java Stream API for transformation
    \item Maps each Quiz entity to QuizSummaryResponse DTO
    \item Returns list of summaries (not full quiz with questions)
\end{itemize}

\subsection{Method: getQuizById}

\begin{lstlisting}[caption=Get Quiz by ID]
public QuizResponse getQuizById(Long id) {
    Quiz quiz = quizRepository.findById(id)
            .orElseThrow(() -> new RuntimeException(''Quiz not found''));
    return toDetailedResponse(quiz);
}
\end{lstlisting}

\textbf{Explanation:}
\begin{itemize}
    \item Fetches quiz by ID
    \item \texttt{Optional.orElseThrow()}: Throws exception if not found
    \item Returns detailed response with all questions and options
    \item Includes correct answers (admin view)
\end{itemize}

\subsection{Method: createQuiz (Part 1)}

\begin{lstlisting}[caption=Create Quiz - Method Start]
@Transactional
public QuizResponse createQuiz(QuizRequest request, 
                              String adminEmail) {
    User admin = userRepository.findByEmail(adminEmail)
            .orElseGet(() -> {
                // Create dummy admin user if not exists
                User newAdmin = new User();
                newAdmin.setEmail(adminEmail);
                newAdmin.setName(''Admin'');
                newAdmin.setPassword(''dummy'');
                newAdmin.setRole(User.Role.ADMIN);
                return userRepository.save(newAdmin);
            });
\end{lstlisting}

\textbf{Explanation:}
\begin{itemize}
    \item \texttt{@Transactional}: Ensures atomic operation
    \item Looks up admin user by email
    \item If user doesn't exist, creates new admin (demo logic)
    \item In production, user should already exist from registration
\end{itemize}

\subsection{Method: createQuiz (Part 2 - Entity Creation)}

\begin{lstlisting}[caption=Create Quiz - Entity Setup]
    Quiz quiz = new Quiz();
    quiz.setTitle(request.title());
    quiz.setDescription(request.description());
    quiz.setDuration(request.duration());
    quiz.setIsActive(request.isActive() != null ? 
                    request.isActive() : true);
    quiz.setCreatedBy(admin);
\end{lstlisting}

\textbf{Explanation:}
\begin{itemize}
    \item Creates new Quiz entity
    \item Sets properties from QuizRequest DTO
    \item Default isActive to true if not provided
    \item Links quiz to admin user
\end{itemize}

\subsection{Method: createQuiz (Part 3 - Questions)}

\begin{lstlisting}[caption=Create Quiz - Questions Loop]
    if (request.questions() != null) {
        for (QuestionRequest qReq : request.questions()) {
            Question question = new Question();
            question.setQuestionText(qReq.questionText());
            question.setType(Question.QuestionType.valueOf(
                qReq.type()));
            question.setPoints(qReq.points() != null ? 
                qReq.points() : 1);
            question.setQuiz(quiz);
\end{lstlisting}

\textbf{Explanation:}
\begin{itemize}
    \item Iterates through questions in request
    \item Creates Question entity for each
    \item Converts string type to QuestionType enum
    \item Defaults points to 1 if not specified
    \item Establishes bidirectional relationship with quiz
\end{itemize}

\subsection{Method: createQuiz (Part 4 - Options)}

\begin{lstlisting}[caption=Create Quiz - Options Loop]
            if (qReq.options() != null) {
                for (OptionRequest oReq : qReq.options()) {
                    Option option = new Option();
                    option.setOptionText(oReq.optionText());
                    option.setIsCorrect(oReq.isCorrect());
                    option.setQuestion(question);
                    question.getOptions().add(option);
                }
            }
            quiz.getQuestions().add(question);
        }
    }
\end{lstlisting}

\textbf{Explanation:}
\begin{itemize}
    \item Nested loop for options within each question
    \item Creates Option entity for each choice
    \item Sets correct answer flag
    \item Establishes question relationship
    \item Adds option to question's list
    \item Adds question to quiz's list
\end{itemize}

