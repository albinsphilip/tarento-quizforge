% Chapter 3 Continued: Models (Part 2)

\section{Question Entity}

\subsection{Purpose}
The \texttt{Question} entity represents individual questions within a quiz. Supports multiple question types including multiple-choice, true/false, and short answer.

\subsection{Complete Source Code}

\begin{lstlisting}[caption=Question.java - Complete Implementation]
package com.quizforge.model;

import jakarta.persistence.*;
import lombok.AllArgsConstructor;
import lombok.Data;
import lombok.NoArgsConstructor;

import java.util.ArrayList;
import java.util.List;

@Entity
@Table(name = ''questions'')
@Data
@NoArgsConstructor
@AllArgsConstructor
public class Question {
    @Id
    @GeneratedValue(strategy = GenerationType.IDENTITY)
    private Long id;

    @ManyToOne(fetch = FetchType.LAZY)
    @JoinColumn(name = ''quiz_id'', nullable = false)
    private Quiz quiz;

    @Column(nullable = false, columnDefinition = ''TEXT'')
    private String questionText;

    @Enumerated(EnumType.STRING)
    @Column(nullable = false)
    private QuestionType type;

    @Column(nullable = false)
    private Integer points = 1;

    @OneToMany(mappedBy = ''question'', cascade = CascadeType.ALL, 
               orphanRemoval = true)
    private List<Option> options = new ArrayList<>();

    public enum QuestionType {
        MULTIPLE_CHOICE, TRUE_FALSE, SHORT_ANSWER
    }
}
\end{lstlisting}

\subsection{Line-by-Line Explanation}

\begin{itemize}[leftmargin=*]
    \item \textbf{Line 21-23: Quiz Relationship}
    \begin{itemize}
        \item \texttt{@ManyToOne}: Many questions belong to one quiz
        \item \texttt{LAZY} fetch: Quiz loaded only when accessed
        \item \texttt{quiz\_id}: Foreign key column in database
        \item Question cannot exist without a quiz (nullable = false)
    \end{itemize}
    
    \item \textbf{Line 25-26: QuestionText Field}
    \begin{itemize}
        \item Stores the actual question text
        \item TEXT type for long questions
        \item Required field
    \end{itemize}
    
    \item \textbf{Line 28-30: Type Field (Enum)}
    \begin{itemize}
        \item \texttt{MULTIPLE\_CHOICE}: Question with multiple options, one correct
        \item \texttt{TRUE\_FALSE}: Binary question with two options
        \item \texttt{SHORT\_ANSWER}: Text-based answer (manual grading)
        \item Stored as string in database for readability
    \end{itemize}
    
    \item \textbf{Line 32-33: Points Field}
    \begin{itemize}
        \item Score value for this question
        \item Defaults to 1 point
        \item Used in calculating total quiz score
        \item Can be customized per question
    \end{itemize}
    
    \item \textbf{Line 35-37: Options Relationship}
    \begin{itemize}
        \item \texttt{@OneToMany}: One question has many options
        \item \texttt{mappedBy = ''question''}: Bidirectional relationship
        \item \texttt{CascadeType.ALL}: All operations cascade to options
        \item \texttt{orphanRemoval}: Delete options if removed from list
        \item Multiple choice questions typically have 4-5 options
    \end{itemize}
    
    \item \textbf{Line 39-41: QuestionType Enum}
    \begin{itemize}
        \item Defines three supported question types
        \item Extensible for future question types
        \item Influences validation logic in frontend
    \end{itemize}
\end{itemize}

\subsection{Database Table Structure}

\begin{table}[h]
\centering
\begin{tabular}{|l|l|l|l|}
\hline
\textbf{Column} & \textbf{Type} & \textbf{Constraints} & \textbf{Description} \\
\hline
id & BIGSERIAL & PRIMARY KEY & Auto-increment ID \\
quiz\_id & BIGINT & NOT NULL, FK & Parent quiz ID \\
question\_text & TEXT & NOT NULL & Question content \\
type & VARCHAR(50) & NOT NULL & Question type \\
points & INTEGER & NOT NULL & Point value \\
\hline
\end{tabular}
\caption{Questions Table Structure}
\end{table}

\section{Option Entity}

\subsection{Purpose}
The \texttt{Option} entity represents answer choices for multiple-choice and true/false questions.

\subsection{Complete Source Code}

\begin{lstlisting}[caption=Option.java - Complete Implementation]
package com.quizforge.model;

import jakarta.persistence.*;
import lombok.AllArgsConstructor;
import lombok.Data;
import lombok.NoArgsConstructor;

@Entity
@Table(name = ''options'')
@Data
@NoArgsConstructor
@AllArgsConstructor
public class Option {
    @Id
    @GeneratedValue(strategy = GenerationType.IDENTITY)
    private Long id;

    @ManyToOne(fetch = FetchType.LAZY)
    @JoinColumn(name = ''question_id'', nullable = false)
    private Question question;

    @Column(nullable = false)
    private String optionText;

    @Column(nullable = false)
    private Boolean isCorrect = false;
}
\end{lstlisting}

\subsection{Line-by-Line Explanation}

\begin{itemize}[leftmargin=*]
    \item \textbf{Line 18-20: Question Relationship}
    \begin{itemize}
        \item \texttt{@ManyToOne}: Many options belong to one question
        \item LAZY loading for performance
        \item Required relationship (nullable = false)
    \end{itemize}
    
    \item \textbf{Line 22-23: OptionText Field}
    \begin{itemize}
        \item The text of the answer choice
        \item Required field
        \item Displayed to candidates during quiz
    \end{itemize}
    
    \item \textbf{Line 25-26: IsCorrect Field}
    \begin{itemize}
        \item Boolean flag indicating correct answer
        \item Defaults to false
        \item Only one option should be true for MULTIPLE\_CHOICE
        \item Used for automatic grading
        \item Hidden from candidates during quiz taking
    \end{itemize}
\end{itemize}

\subsection{Database Table Structure}

\begin{table}[h]
\centering
\begin{tabular}{|l|l|l|l|}
\hline
\textbf{Column} & \textbf{Type} & \textbf{Constraints} & \textbf{Description} \\
\hline
id & BIGSERIAL & PRIMARY KEY & Auto-increment ID \\
question\_id & BIGINT & NOT NULL, FK & Parent question ID \\
option\_text & VARCHAR(255) & NOT NULL & Answer choice text \\
is\_correct & BOOLEAN & NOT NULL & Correct answer flag \\
\hline
\end{tabular}
\caption{Options Table Structure}
\end{table}

\section{QuizAttempt Entity}

\subsection{Purpose}
The \texttt{QuizAttempt} entity tracks when a candidate takes a quiz, storing timing, score, and status information.

\subsection{Complete Source Code}

\begin{lstlisting}[caption=QuizAttempt.java - Complete Implementation]
package com.quizforge.model;

import jakarta.persistence.*;
import lombok.AllArgsConstructor;
import lombok.Data;
import lombok.NoArgsConstructor;

import java.time.LocalDateTime;
import java.util.ArrayList;
import java.util.List;

@Entity
@Table(name = ''quiz_attempts'')
@Data
@NoArgsConstructor
@AllArgsConstructor
public class QuizAttempt {
    @Id
    @GeneratedValue(strategy = GenerationType.IDENTITY)
    private Long id;

    @ManyToOne(fetch = FetchType.LAZY)
    @JoinColumn(name = ''quiz_id'', nullable = false)
    private Quiz quiz;

    @ManyToOne(fetch = FetchType.LAZY)
    @JoinColumn(name = ''user_id'', nullable = false)
    private User user;

    @Column(name = ''started_at'', nullable = false)
    private LocalDateTime startedAt;

    @Column(name = ''submitted_at'')
    private LocalDateTime submittedAt;

    @Column(name = ''score'')
    private Integer score;

    @Column(name = ''total_points'')
    private Integer totalPoints;

    @Enumerated(EnumType.STRING)
    @Column(nullable = false)
    private AttemptStatus status = AttemptStatus.IN_PROGRESS;

    @OneToMany(mappedBy = ''attempt'', cascade = CascadeType.ALL, 
               orphanRemoval = true)
    private List<Answer> answers = new ArrayList<>();

    @PrePersist
    protected void onCreate() {
        if (startedAt == null) {
            startedAt = LocalDateTime.now();
        }
    }

    public enum AttemptStatus {
        IN_PROGRESS, SUBMITTED, EVALUATED
    }
}
\end{lstlisting}

\subsection{Line-by-Line Explanation}

\begin{itemize}[leftmargin=*]
    \item \textbf{Line 22-24: Quiz Relationship}
    \begin{itemize}
        \item Links attempt to specific quiz
        \item One quiz can have many attempts
        \item LAZY loading for performance
    \end{itemize}
    
    \item \textbf{Line 26-28: User Relationship}
    \begin{itemize}
        \item Links attempt to candidate user
        \item One user can have many attempts
        \item Tracks who took the quiz
    \end{itemize}
    
    \item \textbf{Line 30-31: StartedAt Timestamp}
    \begin{itemize}
        \item Records when quiz attempt began
        \item Required field
        \item Used to calculate time remaining
    \end{itemize}
    
    \item \textbf{Line 33-34: SubmittedAt Timestamp}
    \begin{itemize}
        \item Records when candidate submitted answers
        \item Optional (null while in progress)
        \item Used to verify submission within time limit
    \end{itemize}
    
    \item \textbf{Line 36-37: Score Field}
    \begin{itemize}
        \item Points earned by candidate
        \item Null until quiz is evaluated
        \item Calculated by comparing answers with correct options
    \end{itemize}
    
    \item \textbf{Line 39-40: TotalPoints Field}
    \begin{itemize}
        \item Maximum possible score
        \item Sum of all question points
        \item Set when attempt is created
    \end{itemize}
    
    \item \textbf{Line 42-44: Status Field (Enum)}
    \begin{itemize}
        \item \texttt{IN\_PROGRESS}: Quiz started but not submitted
        \item \texttt{SUBMITTED}: Answers submitted, awaiting evaluation
        \item \texttt{EVALUATED}: Graded and score calculated
        \item Defaults to IN\_PROGRESS
    \end{itemize}
    
    \item \textbf{Line 46-48: Answers Relationship}
    \begin{itemize}
        \item One attempt has many answers
        \item Each answer corresponds to one question
        \item CASCADE operations ensure consistency
    \end{itemize}
    
    \item \textbf{Line 50-55: @PrePersist Callback}
    \begin{itemize}
        \item Sets startedAt if not already set
        \item Ensures attempt has start timestamp
        \item Only sets if null (allows manual override)
    \end{itemize}
    
    \item \textbf{Line 57-59: AttemptStatus Enum}
    \begin{itemize}
        \item Three-stage lifecycle
        \item Prevents duplicate submissions
        \item Used in queries to filter attempts
    \end{itemize}
\end{itemize}

\subsection{Database Table Structure}

\begin{table}[h]
\centering
\begin{tabular}{|l|l|l|l|}
\hline
\textbf{Column} & \textbf{Type} & \textbf{Constraints} & \textbf{Description} \\
\hline
id & BIGSERIAL & PRIMARY KEY & Auto-increment ID \\
quiz\_id & BIGINT & NOT NULL, FK & Quiz reference \\
user\_id & BIGINT & NOT NULL, FK & Candidate reference \\
started\_at & TIMESTAMP & NOT NULL & Start time \\
submitted\_at & TIMESTAMP & NULL & Submission time \\
score & INTEGER & NULL & Points earned \\
total\_points & INTEGER & NULL & Max possible score \\
status & VARCHAR(50) & NOT NULL & Attempt status \\
\hline
\end{tabular}
\caption{Quiz Attempts Table Structure}
\end{table}

\section{Answer Entity}

\subsection{Purpose}
The \texttt{Answer} entity stores candidate responses to individual questions, along with grading results.

\subsection{Complete Source Code}

\begin{lstlisting}[caption=Answer.java - Complete Implementation]
package com.quizforge.model;

import jakarta.persistence.*;
import lombok.AllArgsConstructor;
import lombok.Data;
import lombok.NoArgsConstructor;

@Entity
@Table(name = ''answers'')
@Data
@NoArgsConstructor
@AllArgsConstructor
public class Answer {
    @Id
    @GeneratedValue(strategy = GenerationType.IDENTITY)
    private Long id;

    @ManyToOne(fetch = FetchType.LAZY)
    @JoinColumn(name = ''attempt_id'', nullable = false)
    private QuizAttempt attempt;

    @ManyToOne(fetch = FetchType.LAZY)
    @JoinColumn(name = ''question_id'', nullable = false)
    private Question question;

    @ManyToOne(fetch = FetchType.LAZY)
    @JoinColumn(name = ''selected_option_id'')
    private Option selectedOption;

    @Column(columnDefinition = ''TEXT'')
    private String textAnswer;

    @Column(nullable = false)
    private Boolean isCorrect = false;

    @Column
    private Integer pointsEarned = 0;
}
\end{lstlisting}

\subsection{Line-by-Line Explanation}

\begin{itemize}[leftmargin=*]
    \item \textbf{Line 18-20: Attempt Relationship}
    \begin{itemize}
        \item Links answer to quiz attempt
        \item Many answers per attempt
        \item Required relationship
    \end{itemize}
    
    \item \textbf{Line 22-24: Question Relationship}
    \begin{itemize}
        \item Links answer to specific question
        \item Each answer corresponds to one question
        \item Required to know what was answered
    \end{itemize}
    
    \item \textbf{Line 26-28: SelectedOption Relationship}
    \begin{itemize}
        \item For multiple-choice/true-false questions
        \item References chosen option
        \item Optional (nullable) - not used for SHORT\_ANSWER
    \end{itemize}
    
    \item \textbf{Line 30-31: TextAnswer Field}
    \begin{itemize}
        \item For SHORT\_ANSWER question type
        \item Stores text response
        \item Optional (null for multiple choice)
        \item Requires manual grading
    \end{itemize}
    
    \item \textbf{Line 33-34: IsCorrect Field}
    \begin{itemize}
        \item Boolean flag for answer correctness
        \item Defaults to false
        \item Automatically set for multiple-choice
        \item Must be manually set for short answers
    \end{itemize}
    
    \item \textbf{Line 36-37: PointsEarned Field}
    \begin{itemize}
        \item Points awarded for this answer
        \item 0 for incorrect answers
        \item Equal to question.points for correct answers
        \item Used in calculating total score
    \end{itemize}
\end{itemize}

\subsection{Database Table Structure}

\begin{table}[h]
\centering
\begin{tabular}{|l|l|l|l|}
\hline
\textbf{Column} & \textbf{Type} & \textbf{Constraints} & \textbf{Description} \\
\hline
id & BIGSERIAL & PRIMARY KEY & Auto-increment ID \\
attempt\_id & BIGINT & NOT NULL, FK & Attempt reference \\
question\_id & BIGINT & NOT NULL, FK & Question reference \\
selected\_option\_id & BIGINT & NULL, FK & Selected option \\
text\_answer & TEXT & NULL & Text response \\
is\_correct & BOOLEAN & NOT NULL & Correctness flag \\
points\_earned & INTEGER & NULL & Points awarded \\
\hline
\end{tabular}
\caption{Answers Table Structure}
\end{table}

